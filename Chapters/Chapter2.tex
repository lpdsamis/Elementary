\chapter{الحقول و الجسيمــات الأوليــة} % 
\label{Chapter2}
%% \renewcommand{\theequation}{\arabic{chapter}.\arabic{equation}}
%----------------------------------------------------------------------------------------

\section{النسبة الخاصة}
ساد الإعتقاد في ميكانيك الكلاسيكية بأن الزمان و المكان شيئان مستقلان و منفصلان عن بعضهما . الى أن جاءت نظرية النسبية الخاصة و التي أعلن عنها الفيزيائي الشهير البرت انشتاين سنة 1905 و اثبتت أن حركة اﻷجسما توصف بالنسبة لبعضها البعض، و ان للزمان و المكان مفهوم واحد يعبر عنه بمصطلح جديد يسمى " الزمكان ". في آواخر القرن التاسع عشر سيطرت فكرة و جود وسط كوني، سمي باﻷثير يملأ الفراغ، و أن موجات الضوء تنتشر خلال هذا الوسط، أدت هذه الفكرة إلى معضلة سنتعرف عليها من خلال تجربة ميكلسون و مورلي.
\subsection{تجربة ميكلسون و مورلي}
تتمثل التجربة في دراسة أثر حركة اﻷرض عبر الفضاء في سرعة الضوء المقاسة على سطحها. تدور اﻷرض حول نفسها دورة واحدة كل 24 ساعة و بسرعة تقدر بــ $30 km.s^{-1}$                                                                                                                                                                                                                                                                                                                                     
تتلخص تجربة ميكلسون و مورلي في قياس الزمن الذي يستغرقه الضوء عندما ينتشر في اتجاه الرياح اﻷثيرية (المفترضة) ثم في إتجاه متعامد عليها. لتوضيح ذالك نستعرض فكرة " القارب في النهر" حيث ندرس حركة قارب يقوم برحلتين يعود في نهايتها الى نقطة البداية.
\begin{itemize}
	\item الحالة التي يقوم فيها برحلته على طول النهر يكون القارب في رحلة الذهاب مع اتجاه التيار في النهر و تكون محصلة السرعة هي $ (V+ v)$
حيث أن $V$ هي سرعة القارب بالنسبة للماء و $v$ هي سرعة جريان الماء في النهر. حالة العودة يكون القارب يجري ضد التيار و تكون محصلة سرعته هي ($V-v$).  بفرض أن المسافة بين نقطة البداية و نقطة الرجوع هي  $L$  فإن الوقت اللازم للعودة  $t_{1}$  يساوي:
 \begin{equation}
 	t_{1}=\dfrac{L}{V+v}+ \dfrac{L}{V-v} = \dfrac{2L/V}{1-\frac{v^{2}}{V^{2}}}
  \end{equation}
  \item 
الحالة التي يقوم فيها برحلته على عرض النهر نفرض أنه يبدأرحلته من النقطة $a$ ليصل الى النقطة $b$ على عرض النهر مباشرة، نستنتج أنه على القارب أن ينحرف قليلا في بداية خط سيره ضد التيار حتى يعوض ما يحدثه له التيار من إزاحة في إتجاه إنسيابه.
من الشكل \ref{fig:2_1} نجد أن :
\begin{equation}
	V^{2}= V^{'2} + v^{2}
\end{equation}
\begin{equation}
	V^{'} = V\sqrt{1-\frac{v^{2}}{V^{2}}}
\end{equation}
حيث $V^{'}$ هي محصلة السرعة، و عليه فإن الزمن اللازم $t_{2}$ لحركة القارب من النقطة $a$ مرة أخرى يعطى بالعلاقة اﻵتية :
\begin{equation}
	t_{2} = \dfrac{2L/V}{\sqrt{1-v^{2}/V^{2}}}
\end{equation}
\end{itemize} 
لكي تتمثل أمامنا تجربة ميكلسون و مورلي، سوف نستبدل النهر الجاري بالرياح اﻷثيرية $v$ و نستبدل سرعة القارب المتحرك بسرعة الضوء $c$, و عليه فإن الزمن المستغرق من النقطة $a$ إلى النقطة $b$ و العودة مرة أخرى إلى $a$ يعطى من المعادلتين (1 و 4) على التوالي :
\begin{equation}
	t_{1}=\dfrac{2L/c}{1-v^{2}/c^{2}}
\end{equation}
\begin{equation}
	t_{2}= \dfrac{2L/c}{1-v^{2/c^{2}}}
\end{equation}
حيث $v$ تمثل سرعة اﻷثير و $c$ هي سرعة الضوء في الفراغ.
\begin{figure}[h!]
	\centering
	\includegraphics[width=0.6\linewidth, height=0.21\textheight]{"Fig/Fig_II/الشكل 2_1"}
	\caption{القارب الذي يعبر النهر.}
	\label{fig:2_1}
\end{figure}
بالنظر إلى الشكل \ref{fig:-2_2} إذا سقطت حزمة ضوئية من المصدر الضوئي على المرآة نصف المفضضة إنعكست الحزمتان مرة آخرى بواسطة المرآتتين المستويتين و الموضوعتين على بعدين متساويين من المركز. عند وصول الحزمتان إلى المرآة المفضضة نفذ جزء من الشعاع المرتد من المرآة  $C$، بعد ذلك سار الشعاعان ووصلا الكاشف. تمت دراسة نمط التداخل في الكاشف و لم يلاحظ أي تغيير عليه في حالة إدارة الجهاز في عدة إتجاهات، و عليه ليس هناك أي تغير في $t_{1}$ و $t_{2}$. نستخلص من هذه التجربة أن سرعة الضوء ثابتة في جميع الإتجاهات، و أدى هذا إلى إستبعاد فكرة الأثير.
\begin{figure}[h!]
	\centering
	\includegraphics[width=0.6\linewidth, height=0.2\textheight]{"Fig/Fig_II/الشكل 2_2"}
	\caption{تجربة ميكلسون و مورلي.}
	\label{fig:-2_2}
\end{figure}
\subsection{فرضيات النسبية الخاصة}
وضع أينشتاين فرضيتين لمعالجة قصور جاليليو و لتأكيد نتائج ميكلسون و مورلي
\begin{enumerate}
	\item تتخذ القوانين و المبادئ الفيزيائية نفس شكلها الرياضي عندما يعبر عنها في محاور أي من نظم الإسناد القصورية.
	\item سرعة الضوء في الفراغ ثابتة بالنسبة لكل المراقبين بغض النظر عن حالتهم من اسكون أو الحركة بالنسبة لمصدر الضوء.
\end{enumerate}
\subsection{تحويلات جاليليو}
لتحديد موضع جسم في الفضاء نستخدم نظام إسناد ينسب إليه موضع الجسم كانظام الإحدثيات الكارتيزية ($x, y, z$). و عند الحديث عن حركة الجسم يضاف إليه الزمن و عندها تسمى بالإحداثيات الرباعية ($x, y, z, t$). يوضح الشكل\ref{fig:-2_3} نظام الإسناد $S$ الذي نحتاجه لوصف أي حدث من خلال الإحدثيات ($x, y, z, t$). يمكن أيضا وصف نفس الحدث من خلال نظام الإسناد $S'$ بالإحدثيات ($x', y', z', t'$) و الذي يتحرك بسرعة ثابتة $v$ بالنسبة للنظام $S$ في الإتجاه الموجب لمحور ($x$) و أن المحاور المتناظرة تكون متوازية.
\begin{figure}[h!]
	\centering
	\includegraphics[width=0.6\linewidth, height=0.2\textheight]{"Fig/Fig_II/الشكل 2_3"}
	\caption{نظام الإسناد $S$ و $S'$ يتحركان بالنسبة إلى بعضعهما بسرعة ثابتة $v$.}
	\label{fig:-2_3}
\end{figure}
نفرض أنه يوجد مراقب $O$ في النظام $S$ و مراقب آخر $O'$ في النظام $S'$. كلا المراقبين يحمل ساعة لقياس الزمن الذي يبدأ عندما تكون نقطتا اﻷصل للنظمين منطبقتان أي $t=t'=0$. 
يعبر عن العلاقة التي تربط بين إحدثيات النظامين بالعلاقة التالية:
\begin{equation}
	\label{eqn:2-7}
	\begin{array}{ccl}
		x' & = &  x -vt \\
		y' & = & y \\
		z' & = & z \\
		t' & = & t
	\end{array}
\end{equation}
مجموعة المعادلات \eqref{eqn:2-7} تسمى تحويلات جاليليو. يمكن إستخدام هذه التحويلات لوصف اﻷجسام التي تتحرك بسرعات عالية. تعارضت هذه المعادلات مع نظرية الكهرومغناطيسية للضوء، حيث أن القوة المؤثرة على شحنة كهربائية مقاسة بواسطة المراقب $O$ في النظام $S$ تكون مختلفة عما يقيسه المراقب $O'$ في نظام الإسناد$S'$. هذا يعني أن قوانين الكهرومغناطيسية غير محفوظة في نظم الإسناد التي تتحرك بسرعة ثابتة بالنسبة لبعضها البعض، و اتضح أن تحويلات جاليليو لا تتسق مع فرضيتي أنشتاين حيث أن سرعة الضوء في الفراغ لن تظل ثابتة، و عليه باتت الحاجة لنوع آخر من التحويلات ينسجم نظرية انشتاين.
\subsection{تحويلات لـورنتز}
إستخدم اينشتاين فرضى النسبية للحصول على تحويلات لورنتز و انتائج المتحصل عليها أثبتت صحة الفرضيتين. معادلة الموجة الكروية في نظام الإسناد $S$ و $S'$ و الذي يتحرك بسرعة ثابتة $v$ في اتجاه موازي لمحور ($x$). تعطى معادلة الموجى التي سرعتها $c$ بالعلاقة التالية:
 \begin{equation}
 	\label{eqn:2-8}
 	x'^{2} + y'^{2} + z'^{2} - c^{2}t'^{2}=0
 \end{equation}
 أما بالنسبة لمراقب متواجد في النظام  $S$ فإن معادلة الموجة تأخذ التعبير التالي:
  \begin{equation}
  	\label{eqn:2-9}
 	x^{2} + y^{2} + z^{2} - c^{2}t^{2}=0
 \end{equation}
بإستخدام المعادلتين  \eqref{eqn:2-8} و  \eqref{eqn:2-9}  يمكن كتابة :
 \begin{equation}
 	\label{eqn:2-10}
 	x^{2} + y^{2} + z^{2} - c^{2}t^{2}= x'^{2} + y'^{2} + z'^{2} - c^{2}t'^{2}
 \end{equation}
 لتحويل أي من المعادلتين  \eqref{eqn:2-8} و  \eqref{eqn:2-9}  إلى الأخرى يجب معرفة العلاقة بين المتغيرات ($x, y, z$) و ($x', y', z'$). إفترض اينشتاين علاقة خطية بين المتغيرات للحركة في إتجاه  $x$ بحيث أن : 
\begin{equation}
	\label{eqn:2-11}
	\begin{array}{ccl}
	x' = \alpha_{1}.x + \alpha_{2}.t \\
	t' = \beta_{1}.x + \beta_{2}.t	
\end{array}
 \end{equation}
 حيث $\alpha_{1}, \alpha_{2}, \beta_{1}, \beta_{2}$ ثوابت. 
بفرض أنه عند $x'=0$ فإن $x=vt$, بإستخدام هذا الشرط في المعادلة \eqref{eqn:2-11} نجد أن :
\begin{equation}
		\label{eqn:2-12}
	\alpha_{2} = - \alpha_{1}.v
\end{equation}
بتعويض هذه العلاقة في
\eqref{eqn:2-11} 
ينتج : 
\begin{equation}
	\label{eqn:2-13}
	x' = \alpha_{1} (x-vt)
\end{equation}
بإدراج المعادلتين 
\eqref{eqn:2-11} و \eqref{eqn:2-13} في المعادلة \eqref{eqn:2-8} نجد :
\begin{equation*}
\alpha_{1}^{2}(x-vt)^{2} + y^{2}+ z^{2} = c^{2}(\beta_{1}.x + \beta_{2}.t)
\end{equation*}
بعد الترتيب نجد أن :
\begin{equation}
	\label{eqn:2-14}
	(\alpha_{1}^{2}-c^{2}\beta_{1}^{2})x^{2} + y^{2}+ z^{2} = c^{2}(\beta_{2}^{2} -\alpha_{1}^{2}v^{2})t^{2} + 2xt(\alpha_{1}^{2}v-c^{2}\beta_{1}\beta_{2})
\end{equation}
بمقارنة المعاملات للمعادلتين \eqref{eqn:2-9} و \eqref{eqn:2-14} يتضح أن :
\begin{equation}
	\label{eqn:2-15}
	\begin{array}{ccl}
		\alpha_{1}^{2} -c^{2}\beta_{1}^{2} & = &  1 \\
		c^{2}\beta_{2}^{2}-\alpha_{1}^{2}v^{2} & = & c^{2} \\
		\alpha_{1}^{2}v + c^{2}\beta_{1}\beta_{2} & = & 0
	\end{array}
\end{equation}
بحل المعادلات الثلاث اﻷخيرة نحصل على :
\begin{equation}
	\label{eqn:2-16}
	\alpha_{1}=\beta_{2}= \frac{1}{\sqrt{1-\dfrac{v^{2}}{c^{2}}}}
\end{equation}
\begin{equation}
	\label{eqn:2-17}
	\beta_{1}=-\frac{v}{c^{2}}\alpha_{1}= \frac{-\dfrac{v}{c^{2}}}{\sqrt{1-\dfrac{v^{2}}{c^{2}}}}
\end{equation}
تصبح المعادلة \eqref{eqn:2-13} كاﻵتي :
\begin{equation}
	\label{eqn:2-18}
	x' = \frac{x-vt}{\sqrt{1-\dfrac{v^{2}}{c^{2}}}}
\end{equation}
و نحصل على تحويل الزمن بالتعويض في المعادلة \eqref{eqn:2-11} 
\begin{equation}
	\label{eqn:2-19}
	t'= \frac{t-\dfrac{v}{c^{2}}.x}{\sqrt{1-\dfrac{v^{2}}{c^{2}}}}
\end{equation} 
و عليه فإن  مجموعة تحويلات لورنتز للحركة في إتجاه $x$ تكتب كاﻵتي:
\begin{equation}
	\label{eqn:2-20}
	\begin{array}{ccl}
		x' & = &  \frac{x-vt}{\sqrt{1-\dfrac{v^{2}}{c^{2}}}} \\
		y' & = & y \\
		z' & = & z \\
		t' & = & \frac{t-\dfrac{v}{c^{2}}.x}{\sqrt{1-\dfrac{v^{2}}{c^{2}}}}
	\end{array}
\end{equation}
وهذه التحويلات التي إشتقها اينشتاين هي نفسها تحويلات لورنتز حيث 
$c$ 
هي سرعة الضوء في الفراغ 
$c = 3\times 10^{8} m.s^{-1}$. 
و في حالة السرعات المنخفضة بالنسبة إلى سرعة الضوء فإن تحويلاتلورنتز تؤول إلى تحويلات جاليليو.
\subsection{إنكماش الطــول}
نفرض أن لدينا قضيب طوله $L_{0}$ مقاسا في نظام الإسناد $S'$ كما موضح في الشكل \ref{fig:-2_4}
\begin{figure}[h!]
	\centering
	\includegraphics[width=0.6\linewidth, height=0.2\textheight]{"Fig/Fig_II/الشكل 2_4"}
	\caption{إنكماش الطول}
	\label{fig:-2_4}
\end{figure}
المراقب في نظام الإسناد 
$S'$ 
سيجد طول القضيب مساويا لــ :
\begin{equation}
	L_{0} = x'_{2} -x'_{1}
\end{equation}
ماهو طول القضيب بالنسبة لمراقب آخر موجود في النظام الإسناد $S$ ?
نلاحظ أن نظام الإسناد $S'$ يتحرك بسرعة ثابتة $v$ في الإتجاه الموجب لمحور ($x$). بالنسبة للمراقب الموجود في النظام $S$ ان طول القضيب $L$ يعطى بالعلاقة : 
\begin{equation}
	L=x_{2} - x_{1}
\end{equation}
العلاقة بين $L$ و $L_{0}$ تخضع لتحويلات لورنتز.
\begin{equation}
	\begin{aligned}
		x'_{1} & =  \frac{x_{1}-vt}{\sqrt{1-\dfrac{v^{2}}{c^{2}}}} \\
		x'_{2} & =  \frac{x_{2}-vt}{\sqrt{1-\dfrac{v^{2}}{c^{2}}}} \\
		\therefore L_{0}= \frac{x_{2}-vt}{\sqrt{1-\dfrac{v^{2}}{c^{2}}}}- \frac{x_{1}-vt}{\sqrt{1-\dfrac{v^{2}}{c^{2}}}} \\
		& = \frac{x_{2}-x_{1}}{\sqrt{1-\dfrac{v^{2}}{c^{2}}}} = \frac{L}{\sqrt{1-\dfrac{v^{2}}{c^{2}}}}
		\end{aligned}
\end{equation}
أي أن : 
\begin{equation}
	\label{eqn:2-21}
	L=L_{0}\sqrt{1-v^{2}/c^{2}}
\end{equation}
بما أن $v<c$ فإن قيمة الجذر التربيعي في المعادلة
\eqref{eqn:2-21} 
تكون دائما أصغر من الواحد الصحيح، يعني هذا أن طول القضيب مقاسا في نظام الإسناد $S$ يكون دائما أصغر من طوله في النظام $S'$ أي أنه حدث إنكماش نسبي للطول.
\begin{itemize}
	\item مثـــال 1 :
	\item مثـــال 2 :
\end{itemize}
\subsection{تمــدد الزمن}
بالعودة إلى تحويلات لورنتز الموسومة بــ \eqref{eqn:2-7}، نجد أن الزمن نسبي هذا يعني أن الفترة الزمنية الفاصلة بين حدثين تختلف بإختلاف حركة المراقب الذي يقوم بقياسها. نفرض أنه تم رصد حدث في نظام الإسناد 
$S'$ 
و كانت لحظة بدايته بالنسبة لمراقب في هذا الإسناد هي $t'_{1}$ و $t'_{2}$ لحظة نهايته. أما بالنسبة للمراقب في الإسناد $S$ فإنه يقيس $t_{1}$  $t_{2}$، يمكن الحصول على العلالقة بينهما بإستخدام تحويلات لورنتز كاﻵتي :
\begin{equation}
	t_{1} = \dfrac{t'_{1}+\dfrac{v}{c^{2}}.x'}{\sqrt{1-\dfrac{v^{2}}{c^{2}}}}
\end{equation}
و بنفس الطريقة نجد $t_{2}$. 
المراقب في الإسناد 
$S'$  
يحسب الفترة الزمنية لتكون :
\begin{equation}
	t_{0}= t'_{2} - t'_{1}
\end{equation}
أما بالنسبة للمراقب اﻵخر في الإسناد $S$ فسيجد :
\begin{equation}
	t= t_{2} - t_{1}
\end{equation}
بالتعويض عن القيم $t_{1}$  و $t_{2}$ نحصل على قيمة $t$ :
\begin{equation}
	t=\frac{t'_{2}+\dfrac{v}{c^{2}}.x'}{\sqrt{1-\dfrac{v^{2}}{c^{2}}}} - \frac{t'_{1}+\dfrac{v}{c^{2}}.x'}{\sqrt{1-\dfrac{v^{2}}{c^{2}}}} \\
	 = \frac{t'_{2}-t'_{1}}{\sqrt{1-\dfrac{v^{2}}{c^{2}}}}
\end{equation}
أي ان العلاقة بين $t$ و $t_{0}$ تصبح
\begin{equation}
	t = \frac{t_{0}}{\sqrt{1- \dfrac{v^{2}}{c^{2}}}}
\end{equation}
تعرف هذه العلاقة بتمدد الزمن و تعني أن الفترة الزمنية $t$ لساعة يد مثلا تتحرك بالنسبة لمراقب تكون أطول من نفس الفترة الزمنية في حالة سكون الساعة بالنسبة للنفس المراقب.\\
بمقارنة تحويلات جاليليو مع تحويلات لورنتز يظهر الفرق جليا بالنسبة للزمن، حيث افترض جاليليو أن الزمن مطلق أي  أنه نفسه بالنسبة لجميع المراقيب و لا يتعلق بحركة الإسناد. أما تحويلات لورنتز فقد أثبتت غير ذالك، اي ان الزمن نسبي بالنسبة للمراقبين و حركة الإسناد و ليس مطلقا.
\begin{itemize}
	\item مثـــال 1 :
	\item مثـــال 2 :
	\item مثـــال 3 :
\end{itemize}
\section{المقاطع الفعــالة}
\section{معـادلات ديراك}

\section{حقول و جسيمات أولية}




