\chapter{الجسيمات اﻷولية و التفاعلات اﻷساسية} % 
\label{Chapter1}
\renewcommand{\theequation}{\arabic{chapter}.\arabic{equation}}
%----------------------------------------------------------------------------------------

% Define some commands to keep the formatting separated from the content 
\newcommand{\keyword}[1]{\textbf{#1}}
\newcommand{\tabhead}[1]{\textbf{#1}}
\newcommand{\code}[1]{\texttt{#1}}
%\newcommand{\file}[1]{\texttt{\bfseries#1}}
%\newcommand{\option}[1]{\texttt{\itshape#1}}

%----------------------------------------------------------------------------------------

‫في ‬‫ﺍﻟﻨﻤﻮﺫﺝ‬ ‫ﺍﻟﻘﻴﺎﺳﻲ‬ ‫ﻟﻴﺲ‬ ‫ﻫﻨﺎﻙ‬ ‫ﻃﺮﻳﻘﺔ‬ ‫ﻧﻈﺮﻳﺔ‬ ‫ﻟﺘﺤﺪﻳﺪ‬ ‫ﻭﺣﺪﺍﺕ‬ ‫ﺍﻟﺒﻨﺎء‬ ‫ﻟﺬﻟﻚ‬ ‫ﻓﻘﺪ‬ تم‬ تحدﻳﺪﻫﺎ‬ ‫ﺍﻓﺘﺮﺍﺿﻴﺎ‫‪.‬‬ ‫ﻭﻓﻘﺎ لهذا الإفتراض ‫ﻓﺎﻥ‬ ‫ﻟﺒﻨﺎﺕ‬ ‫ﺑﻨﺎء‬ ‫ﺍلمادة‬ ‫ﻫﻲ‬ ‫ﺍﺛﻨﺔ‬ ‫ﻋﺸﺮ‬ ‫ﺟﺴﻴﻤﺎ‬ ‫ﻓﲑﻣﻴﻮﻧﻴﺎ‬ ‫ﺫﺍﺕ‬ ‫ﺳﺒﲔ $s=\dfrac{1}{2}$‬‬ ‫ﳝﻜﻦ لهﺬﻩ‬ ‫ﺍلجسﻴﻤﺎﺕ‬ ‫ﺍﻥ‬ ‫ﺗﺘﻔﺎﻋﻞ‬ ‫ﻣﻊ‬ ‫ﺑﻌﻀﻬﺎ‬ ‫ﺍﻟﺒﻌﺾ‬ ‫ﺑﺘﺒﺎﺩﻝ‬ ‫ﺟﺴﻴﻤﺎﺕ‬ ‫ﺑﻮﺯﻭﻧﻴﺔ‬‫‪.‬‬ ‫ﺗﻨﻘﺴﻢ‬ ‫ﻫﺬﻩ‬ ‫ﺍﻟﻠﺒﻨﺎﺕ‬ الى مجمﻮﻋﺘﲔ‬ ‫ﲢﺘﻮﻱ‬ ‫ﻛﻞ‬ ‫ﻣﻨﻬﺎ‬‫ﻋﻠﻰ‬ ‫ﺳﺘﺔ‬ ‫ﺟﺴﻴﻤﺎﺕ‬ ‫ﻭ‬ ‫ﻛﻞ مجموﻋﺔ‬ ‫ﻣﻘﺴﻤﺔ‬ الى‬ ‫ﺛﻼﺛﺔ‬ ‫ﺍﺟﻴﺎﻝ‬ ‫ﺍﻭ‬ ‫ﻋﻮﺍﺋﻞ‬ يحتوي‬ ‫ﻛﻞ‬ ‫ﺟﻴﻞ‬ ‫ﻋﻠﻰ‬ ‫ﺟﺴﻴﻤﲔ‬‫‪.‬‬ ‫ﺍلمجموﻋﺔ‬ ‫ﺍﻻﻭلى ‫ﻫﻲ ‬‫ﻋﺒﺎﺭﺓ‬ ‫عن ﺳﺘﺔ‬ ‫ﻟﺒﺘﻮﻧﺎﺕ‬ ‫ﻭ‬ ‫ﻫﻲ‬ ‫ﺟﺴﻴﻤﺎﺕ‬ ‫ﻣﻌﺮﻭﻓﺔ‬ تجرﻳﺒﻴﺎ‬ ‫ﺍﻣﺎ‬ ‫ﺍلمجموعة ‫ﺍﻟﺜﺎﻧﻴﺔ‬ ‫ﻓﻬﻲ‬ ‫ﻋﺒﺎﺭﺓ‬ ‫ﻋﻦ‬ ‫ﺳﺘﺔ‬ ‫ﻛﻮﺍﺭﻛﺎﺕ‬ ‫ﻭ‬ ‫ﻫﻲ‬ ‫ﺟﺴﻴﻤﺎﺕ‬ تم ‬‫ﺍﻓﺘﺮﺍﺿﻬﺎ‬ ‫ﻭ‬ لم‬ ‫ﻳﺘﻢ‬ ‫ﺭﺻﺪﻫﺎ‬ ‫ﺑﺸﻜﻞ‬ ‫ﻣﺒﺎﺷﺮ‬ ‫ﻭ‬ ‫ﻫﻲ‬ ‫ﺗﺘﻤﻴﺰ‬ ‫ﺑﻜﻮ‪‬ﺎ‬ تحمل‬ ‫ﺷﺤﻨﺔ‬ ‫ﻛﻬﺮﺑﺎﺋﻴﺔ‬ ‫ﻛﺴﺮﻳﺔ‬ ‫ﻣﻦ‬ ‫ﺷﺤﻨﺔ‬ ‫ﺍﻻﻟﻜﺘﺮﻭﻥ‬‫‪.‬‬ ‫ﺗﺘﻤﻴﺰ‬ ‫ﺟﺴﻴﻤﺎﺕ الجيل ‫ﺍﻻﻭﻝ‬ ‫ﻣﻦ‬ ‫ﻛﻞ‬ ‫ﳎﻤﻮﻋﺔ‬ ‫ﺑﺎ‪‬ﺎ‬ ‫ﻏﲑ‬ ‫ﻗﺎﺑﻠﺔ‬ ‫ﻟﻠﺘﻔﻜﻚ‬ ‫ﺍﻣﺎ‬ ‫ﺟﺴﻴﻤﺎﺕ‬ ‫ﺑﺎﻗﻲ‬ ‫ﺍﻻﺟﻴﺎﻝ‬ ‫ﻓﻘﺪ‬ ‫ﺍﺛﺒﺖ‬ ‫ﲡﺮﻳﺒﻴﺎ‬ أنها‬ ‫ﺗﺘﻔﻜﻚ‬ ‫ﻣﻌﻄﻴﺔ‬ ‫ﺟﺴﻴﻤﺎﺕ ‬‫ﻣﻦ‬ الجيل‬ ‫ﺍﻻﻭﻝ‬‫‪.‬‬
\begin{table}
	\centering
\begin{tabular}{|c|c|c|c|c|c|c|}
	\hline
	& \multicolumn{3}{c|}{اللبتونات} &  \multicolumn{3}{c|}{الكواركات} \\
	\hline
	& الإسم & الشحنة & الكتلة & الإسم  & الكتلة  &  الشحنة \\
	\hline
	 الجيل الأول  & $electron~(e^{-})$  & $-1$  & $0.511$ & $up~(up)$ & $1.5-3$& $\frac{2}{3}$ \\
	 
	 & $\nu_{e}$  & $0$ & $0$ &  $down~(dn)$ & $3-7$ & $-\frac{1}{3}$ \\
	 
	\hline
	 الجيل الثاني  & $muon~(\mu^{-}) $  & $-1$ & $105.7$  &  $charm~(c)$ & $1250\pm 90$ & $\frac{2}{3}$ \\
	 
	 & $\nu_{\mu}$ & $0$ & $0$ &  $strang~(s)$ & $95\pm 25$  & $-\frac{1}{3}$ \\
	 
	\hline
	 الجيل الثالث  & $tau~(\tau^{-})$  & $1777$ & $-1$ &$top~(t)$   & $172\times 10^{3}$  &  $\frac{2}{3}$ \\
	  
	  & $\nu_{\tau} $  & $0$ & $0$ &  $beauty~(b)$ &$4200 \pm 70 $  & $ -\frac{1}{3}$ \\
	\hline
\end{tabular}
\caption{‫ﺍﻟجسيمات اﻷولية في ‫ﺍﻟﻨﻤﻮﺫﺝ‬ ‫ﺍﻟﻘﻴﺎﺳﻲ‬}
\label{tab1}
\end{table}

\section{خصائص الجسيمات اﻷولية}

\subsection{~الكتلة‬:}


تصنف الجسيمات اﻷولية بالنسبة لكتلتها إلى:
\begin{description}
	\item[~الجسيمات عديمة الكتلة:] و تشمل الفوتونات التي تكون وسيطة في التفاعلات الكهرومغناطيسية بحيث كتلتها السكونية 
	$m_{\gamma0 }=0$ و سرعتها
	$v_{\gamma } = 3\times 10^{8}~m/s$
	و هو في حالة حركة دائمة و كتلته الحركية هي 
	$m_{\gamma}=\dfrac{E}{c^{2}}$ 
	و زخمه هو 
	$p_{\gamma}= \dfrac{E}{c}$
	
	\item[الجسيمات الخفيفة (اللبتونات)~:]
	و تشمل الإلكترون و البزوترون و النيوترينو  و الميون و ضديد الميون. وقد افترض ان لكل جسيمة خفيفة عددا كميا يميزها يسمى العدد الكمي اللبتوني 
	($lepton quantum number$)
	ويرمز له بالرمز 
	$l$
و تكون قيمته :
\begin{gather*} 
	l = + 1 ~~~ for ~~~ e^{-} , \mu^{-} , \nu^{-}\\ 
	l = - 1 ~~~ for ~~~ e^{+} , \mu^{+} , \nu^{}
\end{gather*}

و 
$l=0$ للجسيمات المتوسطة الكتلة و الجسيمات الثقيلةالكتلة.

\item[الجسيمات المتوسطة الكتلة (الميزونات)~:]

و تشمل على البايونات
$\pi^{+}, \pi^{0}, \pi^{-}$
و أصل كلمة بايون
($pion$) هو $Pi-meson$ , 
و الكايونات 
$K^{+}, K^{0}, K^{0-}, K^{-}$
و أصل الكلمة كايون 
($Kaon$) هو 
$K-meson$
و ليس لها عددا كميا يميزها.
\item[الجسيمات الثقيلة (الباريونات)~:]
و اشهرها البرتون و ضديده و النيترون و ضديده، و أخف الجسيمات الثقيلة هو البروتون. و لقد أفترض أن لكل جسيمة ثقيلة عددا كميا يميزها يسمى العدد الكمي الباريوني و يرمز له بالرمز 
$B$ و تكون قيمته:

\begin{gather*} 
	B = + 1 ~~~ for ~~~ n , p , \\ 
	B = - 1 ~~~ for ~~~ n{-} , p{-} 
\end{gather*}
و تكون قيمة 
$B =0$
للبتونات و الميزونات، و تشترك الميزونات و الباريونات في كافة أنواع التفاعلات الربعة.

\end{description}

\subsection{الزخم البرمي}
تصنف الجسيمات من حيث زخمها الى قسمين هما:

\begin{description}
	\item[الفيرميونات~:]
هي جسيمات زخمها الزاوي البرمي عددا فرديا مضروبا في 
$\dfrac{1}{2}$
أي أن 
$S_{F}= \dfrac{1}{2}\hbar, \dfrac{3}{2}\hbar, \dfrac{5}{2}\hbar, ...$
و هي تخضع لمبدأ الإستثناء لباولي و الذي ينص على أنه لا يمكن لفيرميونين أن يتواجدا بنفس الحالة الكمية و تكون لهما نفس أعداد الكم الربعة 
($n, l, m_{l}, s$) ، كما أنها تخضع لتوزيع فيرمي ديراك.
	\item[البوزونات~:]
هي جسيمات زخمها الزاوي البرمي يساوي عدد صحيح 
$S_{B}= 0\hbar, 1\hbar, 2\hbar,...$
من هذه الجسيمات الغوتون
$S_{\gamma} =1\hbar$، و البايونات
$S_{\pi} =0\hbar$، و الكايونات 
$S_{K} =0\hbar$
و الببوزونات لا تخضع لمبدأ الإستثناء لباولي و لهذا يمكن وجود أي عدد منها في نفس الحالة الكمية، و هي تخضع لقانون التوزيع لبوز-أينشتاين.
\end{description}

\subsection{ التناظر}
التناظر يشير إلى سلوك دالة الموجة الممثلة لحالة الجسيم عند تغير إشارة الإحداثيات. و هي نوعين: 
\begin{description}
	\item[زوجية:] رياضيا يشاراليها بـ:
	
	\begin{gather*} 
		 P\psi(+x, +y, +z) = P\psi(-x, -y, -z)\\
		 P\psi(+x, +y, +z) = \psi (x, y, z)
	\end{gather*}
أي ان التناظر زوجي $P=1$
\item[فردية:] رياضيا يشاراليها بـ:

\begin{gather*} 
	P\psi(+x, +y, +z) = P\psi(-x, -y, -z)\\
	P\psi(+x, +y, +z) = -\psi (x, y, z)
\end{gather*}
أي ان التناظر فردي $P=-1$
لقد افترض أن كل جسيمة تتفاعل بقوة بارتي معينة، زوجية أو فردية. كما اعتبرت بارتي البرتون و النيوترون زوجية أما البايونات فلها بارتي فردية و هنا نشير إلى أن البارتي للبوزونات و أضدادها تكون فردية. دالة الموجة للبوزونات متناظرة بينما دالة الموجة للفيرميونات تكون غير متناظرة. أما البارتي لجسيم مكون من عدة جسيمات تساوي حاصل ضرب البارتي لمكوناته.
\end{description}
\subsection{ الشحنــة}
معظم الجسيمات اﻷولية ذات شحنة كهربائية و بعضها متعادل. تكون للجسيمة معاكسة لشحنة ضديدها و يشار لضديد الجسيمة بخط فوق الرمز.
\subsection{ الغرابة}

\subsection{ عمر الجسيمة}

\subsection{ ضديد الجسيمة}

\section{القوى اﻷربعة او التفعلات الساسية}

هي القوى التي تسلطها اﻷجسام بعضها على البعض و يوجد في الطبيعة أربعة أنواع القوي او أربعة أنواع من التفاعلات.
 \subsection{ القوى التثاقلية}
هي قوى جذب بين اﻷجسام التي لها كتل، استنادا إلى قانون الجذب العام لنيوتن:
\begin{equation}
	F=G.\dfrac{m_{1}.m_{2}}{r^{2}}
\end{equation}
و تتبادل هذه القوى جسيمات تسمى الكرافيتون ($gravitons$).
القوى التثاقلية هي أضعف القوى، و نسبتها إلى القوى النووية هي
$10^{-39}$ 
و ذات مدى أطول، فهي تتحكم بحركة الكواكب و النجوم الساقطة و تأثيرها بالنسبة للجسيمات الولية مهمل لصغر كتل هذه الجسيمات.
 \subsection{ القوى الكهرومغناطيسية}
 هي قوى تجاذب أو تنافر بين الشحنات الكهربائية، بتبادل كمات من الطاقة و الزخم الكهرومغناطيسي و التي تسمى الفوتونات ($photons$).
 نسبة القوى الكهرومغناطيسية إلى القوى النووية هي $10^{-2}$، و مداها غير محدد.
 \subsection{ القوى النووية القوية}
هي القوى التي تربط مكونات النواة (البرتونات و النترونات) مع بعضها، و هي التي تتحكم بالتفاعلات النووية بين الجسيمات اﻷولية ذات الطاقة العالية. و تتصف هذه القوى بما يأتي:
\begin{enumerate}
	\item مداها قصيرة جدا بحدود $10^{-15}m$، وان زادت المسافة عن هذا المدى و لو قليلا ﻷصبحت القوى النووية بينهما مهملة تقريبا.
	
	\item القوى النووية لا تعتمد على الشحنة، فالقوة بين برتون و برتون هي نفسها بين نيوترون و نيوترون أو بين نيوترون وبروتون.
\begin{gather*} 
	F_{pp} \approx F_{nn} \approx F_{pn}
\end{gather*}

	\item تعتمد على محصلة الزخم الزاوي البرمي للجسيمات، فالقوة بين برتونين مثلا برميهما بنفس الاتجاه هي غير القوى التي بينهما عندما يكون برميهما متعاكسين.
	
	\item 
في عام 1935 إفترض يوكاوا ($Yukawa$) المعادلة اﻵتية لتمثيل الطاقة الكامنة النووية:
\begin{equation}
	E_{P}(r) = \pm \dfrac{E_{0}.r_{0}.e^{(-\frac{r}{r_{0}})}}{r}
\end{equation}
حيث $r_{0}$ هو المدى، $E_{0}$ 
كمية ثابتة و 
$r$
المسافة عن مركز النواة. كما افترض يوكاوا ان القوى النووية تتبادل مع بعضها جسيمات أسماها البايونات. تسمى الجسيمات التي تؤثر على بعضها بقوى نووية بـالهادرونات و تشمل الجسيمات الثقيلة و الجسيمات متوسطة الكتلة مثل البايونات و الكايونات، أما الجسيمات الخفيفة كالإلكترونات و الميون و النيوترينيو فلا يمكن أن تشترك في التفاعلات القوية.
\end{enumerate}

 \subsection{ القوى الضعيفة}
هي القوى التي تتحكم بتفاعل الجسيمات الخفيفة و بتحلل الجسيمات اﻷولية، اي ان تحلل أي جسيمة أولية هو تفاعل ضعيف او ينجز بتأثير قوة ضعيفة. زمن تأثير القوة الضعية بحدود $10^{-10}s$ و هو أطول بكثير من زمن تأثير القوى النووية القوية و القوى الكهرومغناطيسية. و تتبادل القوى النووية الضعيفة جسيمات تسمى $W-particles$ 
كتلتها بحدود $800 Mev$.
